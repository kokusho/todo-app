\documentclass[12pt,titlepage]{article}

\usepackage[ngerman]{babel}
\usepackage[latin1]{inputenc}
\usepackage{color}
\usepackage[a4paper,lmargin={4cm},rmargin={2cm},
tmargin={2.5cm},bmargin = {2.5cm}]{geometry}
\usepackage{amssymb}
\usepackage{amsthm}
\usepackage{graphicx}
\linespread{1.25}

\begin{document}

%%% Titelseite %%%
\begin{titlepage}
\title{Projekt - Report \\ Webtechnologien 2}
\date{25.06.2020}
\author{Malte Eisen}
\maketitle
\end{titlepage}

%%% Inhaltsverzeichnis %%%
\newpage
\tableofcontents
\newpage
%%% Hauptteil %%%

%%% Einleitung %%%
\section{Einleitung}
Im Rahmen der Veranstaltung "Webtechnologien 2" im Sommersemester 2020 sollte ein Software-Projekt angefertigt werden. Ziel dieses Projekts war es, die folgenden in der Vorlesung vorgestellten technischen Konzepte und Tools anzuwenden und zu \"uben:\\
- Maven als Build-Management-Tool\\
- die JPA-Schnittstelle zum \"Uberf\"uhren von objektorientierten Daten in eine relationale Datenbank\\
- das REST-Paradigma\\
- das Angular-Framework zum Entwickeln des Front-Ends\\
- ALEX zum Lernen und Testen der Anwendung\\
\\
Dabei sollte eine Webanwendung entwickelt werden, welche ToDo-Notes verwaltet. Diese Notes sollen erstellt, angezeigt, ge\"andert und gel\"oscht werden k\"onnen. Zus\"atzlich soll ein User-Management implementiert werden, welches Registrieren, Einloggen und die Verkn\"upfung zwischen Notes und Usern erm\"oglicht. 

 
%%% Anwendungsgebiet %%%
\section{Anwendungsgebiet}

%%% Entwicklungsprozess %%%
\section{Entwicklungsprozess}

\subsection{Design-Entscheidungen}
Zum Speichern der Daten entschieden wir uns f\"ur eine MySQLDatenbank. 
\subsection{Schwierigkeiten w\"ahrend des Entwicklungsprozess}

\subsection{ALEX}

%%% Ergebnisse %%%
\section{Ergebnisse}










\end{document}